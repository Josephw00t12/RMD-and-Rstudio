% Options for packages loaded elsewhere
\PassOptionsToPackage{unicode}{hyperref}
\PassOptionsToPackage{hyphens}{url}
%
\documentclass[
]{article}
\usepackage{amsmath,amssymb}
\usepackage{iftex}
\ifPDFTeX
  \usepackage[T1]{fontenc}
  \usepackage[utf8]{inputenc}
  \usepackage{textcomp} % provide euro and other symbols
\else % if luatex or xetex
  \usepackage{unicode-math} % this also loads fontspec
  \defaultfontfeatures{Scale=MatchLowercase}
  \defaultfontfeatures[\rmfamily]{Ligatures=TeX,Scale=1}
\fi
\usepackage{lmodern}
\ifPDFTeX\else
  % xetex/luatex font selection
\fi
% Use upquote if available, for straight quotes in verbatim environments
\IfFileExists{upquote.sty}{\usepackage{upquote}}{}
\IfFileExists{microtype.sty}{% use microtype if available
  \usepackage[]{microtype}
  \UseMicrotypeSet[protrusion]{basicmath} % disable protrusion for tt fonts
}{}
\makeatletter
\@ifundefined{KOMAClassName}{% if non-KOMA class
  \IfFileExists{parskip.sty}{%
    \usepackage{parskip}
  }{% else
    \setlength{\parindent}{0pt}
    \setlength{\parskip}{6pt plus 2pt minus 1pt}}
}{% if KOMA class
  \KOMAoptions{parskip=half}}
\makeatother
\usepackage{xcolor}
\usepackage[margin=1in]{geometry}
\usepackage{color}
\usepackage{fancyvrb}
\newcommand{\VerbBar}{|}
\newcommand{\VERB}{\Verb[commandchars=\\\{\}]}
\DefineVerbatimEnvironment{Highlighting}{Verbatim}{commandchars=\\\{\}}
% Add ',fontsize=\small' for more characters per line
\usepackage{framed}
\definecolor{shadecolor}{RGB}{248,248,248}
\newenvironment{Shaded}{\begin{snugshade}}{\end{snugshade}}
\newcommand{\AlertTok}[1]{\textcolor[rgb]{0.94,0.16,0.16}{#1}}
\newcommand{\AnnotationTok}[1]{\textcolor[rgb]{0.56,0.35,0.01}{\textbf{\textit{#1}}}}
\newcommand{\AttributeTok}[1]{\textcolor[rgb]{0.13,0.29,0.53}{#1}}
\newcommand{\BaseNTok}[1]{\textcolor[rgb]{0.00,0.00,0.81}{#1}}
\newcommand{\BuiltInTok}[1]{#1}
\newcommand{\CharTok}[1]{\textcolor[rgb]{0.31,0.60,0.02}{#1}}
\newcommand{\CommentTok}[1]{\textcolor[rgb]{0.56,0.35,0.01}{\textit{#1}}}
\newcommand{\CommentVarTok}[1]{\textcolor[rgb]{0.56,0.35,0.01}{\textbf{\textit{#1}}}}
\newcommand{\ConstantTok}[1]{\textcolor[rgb]{0.56,0.35,0.01}{#1}}
\newcommand{\ControlFlowTok}[1]{\textcolor[rgb]{0.13,0.29,0.53}{\textbf{#1}}}
\newcommand{\DataTypeTok}[1]{\textcolor[rgb]{0.13,0.29,0.53}{#1}}
\newcommand{\DecValTok}[1]{\textcolor[rgb]{0.00,0.00,0.81}{#1}}
\newcommand{\DocumentationTok}[1]{\textcolor[rgb]{0.56,0.35,0.01}{\textbf{\textit{#1}}}}
\newcommand{\ErrorTok}[1]{\textcolor[rgb]{0.64,0.00,0.00}{\textbf{#1}}}
\newcommand{\ExtensionTok}[1]{#1}
\newcommand{\FloatTok}[1]{\textcolor[rgb]{0.00,0.00,0.81}{#1}}
\newcommand{\FunctionTok}[1]{\textcolor[rgb]{0.13,0.29,0.53}{\textbf{#1}}}
\newcommand{\ImportTok}[1]{#1}
\newcommand{\InformationTok}[1]{\textcolor[rgb]{0.56,0.35,0.01}{\textbf{\textit{#1}}}}
\newcommand{\KeywordTok}[1]{\textcolor[rgb]{0.13,0.29,0.53}{\textbf{#1}}}
\newcommand{\NormalTok}[1]{#1}
\newcommand{\OperatorTok}[1]{\textcolor[rgb]{0.81,0.36,0.00}{\textbf{#1}}}
\newcommand{\OtherTok}[1]{\textcolor[rgb]{0.56,0.35,0.01}{#1}}
\newcommand{\PreprocessorTok}[1]{\textcolor[rgb]{0.56,0.35,0.01}{\textit{#1}}}
\newcommand{\RegionMarkerTok}[1]{#1}
\newcommand{\SpecialCharTok}[1]{\textcolor[rgb]{0.81,0.36,0.00}{\textbf{#1}}}
\newcommand{\SpecialStringTok}[1]{\textcolor[rgb]{0.31,0.60,0.02}{#1}}
\newcommand{\StringTok}[1]{\textcolor[rgb]{0.31,0.60,0.02}{#1}}
\newcommand{\VariableTok}[1]{\textcolor[rgb]{0.00,0.00,0.00}{#1}}
\newcommand{\VerbatimStringTok}[1]{\textcolor[rgb]{0.31,0.60,0.02}{#1}}
\newcommand{\WarningTok}[1]{\textcolor[rgb]{0.56,0.35,0.01}{\textbf{\textit{#1}}}}
\usepackage{graphicx}
\makeatletter
\def\maxwidth{\ifdim\Gin@nat@width>\linewidth\linewidth\else\Gin@nat@width\fi}
\def\maxheight{\ifdim\Gin@nat@height>\textheight\textheight\else\Gin@nat@height\fi}
\makeatother
% Scale images if necessary, so that they will not overflow the page
% margins by default, and it is still possible to overwrite the defaults
% using explicit options in \includegraphics[width, height, ...]{}
\setkeys{Gin}{width=\maxwidth,height=\maxheight,keepaspectratio}
% Set default figure placement to htbp
\makeatletter
\def\fps@figure{htbp}
\makeatother
\setlength{\emergencystretch}{3em} % prevent overfull lines
\providecommand{\tightlist}{%
  \setlength{\itemsep}{0pt}\setlength{\parskip}{0pt}}
\setcounter{secnumdepth}{-\maxdimen} % remove section numbering
\usepackage{booktabs}
\usepackage{caption}
\usepackage{longtable}
\usepackage{colortbl}
\usepackage{array}
\usepackage{anyfontsize}
\usepackage{multirow}
\ifLuaTeX
  \usepackage{selnolig}  % disable illegal ligatures
\fi
\usepackage{bookmark}
\IfFileExists{xurl.sty}{\usepackage{xurl}}{} % add URL line breaks if available
\urlstyle{same}
\hypersetup{
  pdftitle={FA-4 R 6.1},
  pdfauthor={Espiritu, Joseph Raphael M.},
  hidelinks,
  pdfcreator={LaTeX via pandoc}}

\title{FA-4 R 6.1}
\author{Espiritu, Joseph Raphael M.}
\date{2025-03-03}

\begin{document}
\maketitle

\subsubsection{\texorpdfstring{\textbf{Exercises
6.1}}{Exercises 6.1}}\label{exercises-6.1}

\paragraph{\texorpdfstring{\textbf{\emph{5.}} A geospatial analysis
system has four sensors supplying images. The percent- age of images
supplied by each sensor and the percentage of images relevant to a query
are shown in the following
table.}{5. A geospatial analysis system has four sensors supplying images. The percent- age of images supplied by each sensor and the percentage of images relevant to a query are shown in the following table.}}\label{a-geospatial-analysis-system-has-four-sensors-supplying-images.-the-percent--age-of-images-supplied-by-each-sensor-and-the-percentage-of-images-relevant-to-a-query-are-shown-in-the-following-table.}

\begin{table}[!t]
\caption*{
{\large Sensor Data} \\ 
{\small Percentage of Images Supplied and Relevant Images}
} 
\fontsize{12.0pt}{14.4pt}\selectfont
\begin{tabular*}{\linewidth}{@{\extracolsep{\fill}}rrr}
\toprule
Sensor & Images Supplied & Relevant Images \\ 
\midrule\addlinespace[2.5pt]
1 & 15\% & 50\% \\ 
2 & 20\% & 60\% \\ 
3 & 25\% & 80\% \\ 
4 & 40\% & 85\% \\ 
\bottomrule
\end{tabular*}
\end{table}

What is the overall Percentage of the Relevant Images?

This is one Equation to solve it

\[
P_{\text{overall}} = \frac{\sum (P_{\text{supplied}, i} \times P_{\text{relevant}, i})}{\sum P_{\text{supplied}, i}}
\]

\begin{Shaded}
\begin{Highlighting}[]
\NormalTok{percentSupplied }\OtherTok{\textless{}{-}} \FunctionTok{c}\NormalTok{(}\DecValTok{15}\NormalTok{, }\DecValTok{20}\NormalTok{, }\DecValTok{25}\NormalTok{, }\DecValTok{40}\NormalTok{)}
\NormalTok{percentRelevant }\OtherTok{\textless{}{-}} \FunctionTok{c}\NormalTok{(}\DecValTok{50}\NormalTok{, }\DecValTok{60}\NormalTok{, }\DecValTok{80}\NormalTok{, }\DecValTok{85}\NormalTok{)}

\NormalTok{overallPercentage }\OtherTok{\textless{}{-}} \FunctionTok{sum}\NormalTok{(percentSupplied }\SpecialCharTok{*}\NormalTok{ percentRelevant) }\SpecialCharTok{/} \FunctionTok{sum}\NormalTok{(percentSupplied)}

\FunctionTok{cat}\NormalTok{(}\StringTok{"Overall percentage of the Relevant images:"}\NormalTok{, overallPercentage, }\StringTok{"\%}\SpecialCharTok{\textbackslash{}n}\StringTok{"}\NormalTok{)}
\end{Highlighting}
\end{Shaded}

\begin{verbatim}
## Overall percentage of the Relevant images: 73.5 %
\end{verbatim}

\paragraph{\texorpdfstring{\textbf{\emph{6.}} A fair coin is tossed
twice.}{6. A fair coin is tossed twice.}}\label{a-fair-coin-is-tossed-twice.}

Let E be the event that both tosses have the same outcome, that is, E1 =
(HH, TT). Let E2 be the event that the first toss is a head, that is, E2
= (HH, HT). Let E3 be the event that the second toss is a head, that is,
E3 = (TH, HH). Show that E1, E2, and E3 are pairwise independent but not
mutually independent.

\begin{table}[!t]
\caption*{
{\large Coin Toss Outcomes}
} 
\fontsize{12.0pt}{14.4pt}\selectfont
\begin{tabular*}{\linewidth}{@{\extracolsep{\fill}}lll}
\toprule
Result & Heads & Tails \\ 
\midrule\addlinespace[2.5pt]
Heads & HH & HT \\ 
Tails & TH & TT \\ 
\bottomrule
\end{tabular*}
\end{table}

\[
(E_1) = \{ \text{HH}, \text{TT} \} \\
\text{E1 is the Event both results are the Same} \\
(E_2) = \{ \text{HH}, \text{HT} \} \\
\text{E2 is the Event first results are the Heads} \\
(E_3) = \{ \text{HH}, \text{TH} \} \\
\text{E3 is the Event second results are the Heads} \\
P(E_1) = P(E_2) = P(E_3) = \frac{1}{2} \\
\]

\paragraph{To show PAIRWISE INDEPENDENCE we can just show their
Union/Probabilities
Together}\label{to-show-pairwise-independence-we-can-just-show-their-unionprobabilities-together}

\[
P(E_i \cap E_j) = P(E_i) \times P(E_j)
\]

For \(E_1\) and \(E_2\):

\[
P(E_1 \cap E_2) = P(\text{HH}) = \frac{1}{4}
\] \[
P(E_1) \times P(E_2) = \frac{1}{2} \times \frac{1}{2} = \frac{1}{4}
\]

Similarly:

\[
P(E_1 \cap E_3) = P(\text{HH}) = \frac{1}{4}
\] \[
P(E_1) \times P(E_3) = \frac{1}{2} \times \frac{1}{2} = \frac{1}{4}
\] \[
P(E_2 \cap E_3) = P(\text{HH}) = \frac{1}{4}
\] \[
P(E_2) \times P(E_3) = \frac{1}{2} \times \frac{1}{2} = \frac{1}{4}
\]

\paragraph{These prove their PAIRWISE INDEPENDENCE, but now to show they
are NOT MUTUALLY
INDEPENDENT}\label{these-prove-their-pairwise-independence-but-now-to-show-they-are-not-mutually-independent}

\paragraph{We can apply the same principle of multiplying the
probabilities and their unions, and we will see the
problem}\label{we-can-apply-the-same-principle-of-multiplying-the-probabilities-and-their-unions-and-we-will-see-the-problem}

\[
P(E_1 \cap E_2 \cap E_3) = P(E_1) \times P(E_2) \times P(E_3)
\]

\[
P(E_1 \cap E_2 \cap E_3) = P(\text{HH}) = \frac{1}{4}
\]

\[
P(E_1) \times P(E_2) \times P(E_3) = \frac{1}{2} \times \frac{1}{2} \times \frac{1}{2} = \frac{1}{8}
\]

\paragraph{\texorpdfstring{Hence, they are not mutually independent
because \(E_2\) and \(E_3\) themselves don't carry enough information to
help form the probability of
\(E_1\).}{Hence, they are not mutually independent because E\_2 and E\_3 themselves don't carry enough information to help form the probability of E\_1.}}\label{hence-they-are-not-mutually-independent-because-e_2-and-e_3-themselves-dont-carry-enough-information-to-help-form-the-probability-of-e_1.}

\paragraph{\texorpdfstring{However, having the union of \(E_2\) and
\(E_3\) changes the probability of \(E_1\) happening, making it
guaranteed.}{However, having the union of E\_2 and E\_3 changes the probability of E\_1 happening, making it guaranteed.}}\label{however-having-the-union-of-e_2-and-e_3-changes-the-probability-of-e_1-happening-making-it-guaranteed.}

\end{document}
